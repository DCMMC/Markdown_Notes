\chapter{大数据分析}
\label{chap:big_data_analysis}

\chapternote{两次平时小作业:$25\%\times2$, 一次课程大作业(机器翻译或推荐系统): $50\%$}{吴志勇老师@Winter 2020}

\begin{learningobjectives}
	\item 绪论
	\item 数据统计分析数学基础
	\item 分析与处理方法
	\item 分布式与并行计算
	\item 前沿
\end{learningobjectives}

\dependencies{无}

\section{数据抽样和假设检验}

例:产品品控的检验:随机抽样来求次品率。

统计推断:Statistical inference is the act of generalizing from a sample (抽样) to a population (总体) with calculated degree of certainty (置信度).
e.g. $\overline{x} \rightarrow \mu$.

Metric: Precision and Reliability.

Sampling Distribution of a Mean ($\overline{x}$) (SDM).

Central Limit Theorem, unbiasedness (无偏估计), seqaure root law (估计方差: $\sigma_{\overline{x}} = \frac{\sigma_{x_i}}{\sqrt(n)}$).

假设检验:

小概率推断: $1 < \alpha \le 0.05$

$\alpha$ 显著性水平, $Z$ 统计量, 统计假设 $H_0,H_1$.
