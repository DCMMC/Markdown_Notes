% !TeX encoding = UTF-8
% !TeX program = xelatex
% !TeX spellcheck = en_US

\documentclass[degree=project,degree-type=project,cjk-font=noto]{thuthesis}
\usepackage{mathtools}
\usepackage{tikz}
\usetikzlibrary{shapes,arrows}
\usepackage[autosize]{dot2texi}
% Syntax Highlighting in LaTeX, need pygments
% Must build with xelatex -shell-escape -enable-8bit-chars.
\usepackage{minted}
% https://tex.stackexchange.com/a/112573
\usepackage{tcolorbox}
\usepackage{etoolbox}
\BeforeBeginEnvironment{minted}{\begin{tcolorbox}}%
\AfterEndEnvironment{minted}{\end{tcolorbox}}%
% color for minted
\definecolor{friendlybg}{HTML}{f0f0f0}


% 论文基本配置,加载宏包等全局配置
\thusetup{
    output = electronic,
    title  = {作业三},
    author  = {肖文韬},
    studentid = {2020214245},
    major = {电子信息(计算机技术)},
    email = {xwt20@mails.tsinghua.edu.cn},
    course = {密码学与网络安全},
    include-spine = false,
}


\usepackage{float}
\usepackage[sort]{natbib}
\bibliographystyle{thuthesis-numeric}
\graphicspath{{figures/}}


\setlist[enumerate,1]{label=\arabic*.}
\setlist[enumerate,2]{label=(\alph*)}
\setlist[enumerate,3]{label=\roman*.}
\setlist[enumerate,4]{label=\greek*}


\begin{document}

% 封面
\maketitle

\frontmatter
% % !TeX root = ../thuthesis-example.tex

% 中英文摘要和关键字

\begin{abstract}
  论文的摘要是对论文研究内容和成果的高度概括。摘要应对论文所研究的问题及其研究目
  的进行描述,对研究方法和过程进行简单介绍,对研究成果和所得结论进行概括。摘要应
  具有独立性和自明性,其内容应包含与论文全文同等量的主要信息。使读者即使不阅读全
  文,通过摘要就能了解论文的总体内容和主要成果。

  论文摘要的书写应力求精确、简明。切忌写成对论文书写内容进行提要的形式,尤其要避
  免“第 1 章……;第 2 章……;……”这种或类似的陈述方式。

  本文介绍清华大学论文模板 \thuthesis{} 的使用方法。本模板符合学校的本科、硕士、
  博士论文格式要求。

  本文的创新点主要有:
  \begin{itemize}
    \item 用例子来解释模板的使用方法;
    \item 用废话来填充无关紧要的部分;
    \item 一边学习摸索一边编写新代码。
  \end{itemize}

  关键词是为了文献标引工作、用以表示全文主要内容信息的单词或术语。关键词不超过 5
  个,每个关键词中间用分号分隔。(模板作者注:关键词分隔符不用考虑,模板会自动处
  理。英文关键词同理。)

  % 关键词用“英文逗号”分隔
  \thusetup{
    keywords = {TeX, LaTeX, CJK, 模板, 论文},
  }
\end{abstract}

\begin{abstract*}
  An abstract of a dissertation is a summary and extraction of research work
  and contributions. Included in an abstract should be description of research
  topic and research objective, brief introduction to methodology and research
  process, and summarization of conclusion and contributions of the
  research. An abstract should be characterized by independence and clarity and
  carry identical information with the dissertation. It should be such that the
  general idea and major contributions of the dissertation are conveyed without
  reading the dissertation.

  An abstract should be concise and to the point. It is a misunderstanding to
  make an abstract an outline of the dissertation and words ``the first
  chapter'', ``the second chapter'' and the like should be avoided in the
  abstract.

  Key words are terms used in a dissertation for indexing, reflecting core
  information of the dissertation. An abstract may contain a maximum of 5 key
  words, with semi-colons used in between to separate one another.

  \thusetup{
    keywords* = {TeX, LaTeX, CJK, template, thesis},
  }
\end{abstract*}


% 目录
% \tableofcontents

% 插图和附表清单
% \listoffiguresandtables
% \listoffigures           % 插图清单

% 正文部分
\mainmatter

\chapter{作业内容}

\begin{enumerate}
  \setlength{\itemsep}{3\parskip}
  \item 在RSA中,如果只知道公钥e和模数N,不知道p、q,求的私钥d的计算复杂度是多少或多大数量集?如果已知p、q呢?请给出证明。
  \newline
  {\heiti 解:}

  如果n可以被质因数分解,d就可以算出,也就意味着私钥被破解。可是,大整数的因数分解,是一件非常困难的事情。暴力枚举的话就是 $O(N)$,不过因为 $N$ 特别大,例如目前常用的 $2048$ bits 的 RSA,$N$ 能达到 $2^{2048} \approx 10^{616}$ 的规模。根据维基百科Prime number theorem条目,质数在自然数中的分布大概是 $\pi(M) \sim \frac{N}{\log_e (N)}$,所以暴力枚举质数的复杂度大概为 $O(\frac{N}{\log_e (N)}$。

  如果已知 p、q,我们需要使用扩展欧几里得算法来计算 $de \equiv 1 (\text{ mod } \phi(N))$ ($e, \phi(N) = (p-1)(q-1)$ 均已知)中的 $d$,所以复杂度为 $O(\log e \log \phi(N))$.

\item 在RSA中:1)给出n=3937和e=17,求私钥d。2)给出n=253和e=3,求私钥d,并对明文m=165进行加密求出密文c。
\newline
{\heiti 解:}
\newline
1) 由 $de \equiv 1 (\text{ mod } \phi(n))$,等价于 $de + k\phi(n) = 1 = \text{gcd}(d, e)$ (关于 $d$ 和 $k$ 的二元一次方程)。其中 $\phi(n) = (p-1)(q-1), n=pq$ ($pq$ 为两个质数),可以通过暴力枚举得到 $3937 = 31 \times 127$,故 $\phi(n) = 30 \times 126 = 3780$。该式可以用扩展欧几里得算法求解,得到 $d = 667, k = -3$。

2) 同理,使用扩展欧几里得算法求解 $de + k\phi(n) = 1$ (其中 $\phi(n) = \phi(253) = \phi(11\times23) = 220$),得到 $d = 73$。对于加密过程,即求 $c = m^e \text{ mod } n$,又因为 $(a \times b) \text{ mod } p = (a \text{ mod } p) \times (b \text{ mod } p)$。故 $c = 165^3 \text{ mod } 253 = 110$。

\item 设 $E$ 是由 $y^2 \equiv x^3 + x + 6 \text{ mod } 11$ 所确定的有限域 $Z_{11}$ 上的椭圆曲线,设基点 $P=(2,7)$, 保密的私钥 $d = 7$。1)计算 $2P = P + P$ 2) 若公钥 $Q = 7P = (7,2)$,假设明文 $m=(5,6)$,计算对应的密文。

{\heiti 解:}

\textbf{因为课上讲的比较少,可以不做。}

\item 简述杂凑函数有哪些功能?应满足哪些条件?

{\heiti 解:}

杂凑函数即哈希函数(hash),是一种将任意数据压缩成摘要(哈希值)的单向函数。哈希值通常用一个短的随机字母和数字组成的字符串来代表。

一般来说需要满足下列条件:

\begin{enumerate}
\item Hash可用于任意大小的数据块。
\item 输出为固定长度的消息摘要。
\item 单向性。给定一个输入M,一定有一个h与其对应,满足H(M)=h,反之,则不行,算法操作是不可逆的。
\item 抗碰撞性。给定一个M,要找到一个M’满足是不可H(M)=H(M’)是不可能的。即不能同时找到两个不同的输入使其输出结果完全一致。
\item 低复杂性:算法具有运算的低复杂性。
\end{enumerate}

\item 什么是陷门单向函数?陷门单向函数有何特点?如何将其应用于公钥密码体制中?

{\heiti 解:}

一个函数是单向陷门函数,是指该函数正向计算是易于计算的,但求它的逆是不可行的,除非再已知某些附加信息。当附加信息给定后,求逆可在多项式时间完成(NP问题)。

研究公钥密码算法就是要找出合适的陷门单向函数。
例如背包密码体制,RSA算法都是利用了逆运算在不附加信息时计算不可行(NP问题)的基础上实现PKI的。
公钥即是陷门单向函数的正向计算过程,计算比较简单,而且可以公开。
而在不知道私钥的情况下逆向计算就是单向陷门函数的逆向计算,其计算不可达性保证了加密算法的安全性。
而私钥就是附加信息,在知道私钥的情况下,单向陷门函数的逆向计算也变成了很容易计算的P 问题。


\item 思考题: 如何衡量一个密码系统的安全性?Diffie-Hellman 算法的安全性如何?

{\heiti 解:}

一个密码系统的安全性主要与这些方面有关:

\begin{enumerate}
  \item \textbf{无条件安全性}。即使密码分析者拥有无限的计算资源和密文,都没有足够的信息恢复出明文,那么这个算法就具有无条件安全性。香农曾经证明了一次一密乱码本(one-timepad)是不可破解的,此种情形下,密钥流是完全随机的、与明文相同长度的比特串,即使给出无限多的资源仍然不可破。虽然其具有理论上的绝对安全性,但考虑到密钥传输的代价,它又是不实用的。
  \item \textbf{计算安全性}。在实际中,无条件安全的系统是不存在的,我们通常所说的算法安全性,就是指算法的计算安全性。如果算法用现在或者将来的可用资源都不能破译,那么,这个算法被认为是计算安全的。在实际应用系统中,当破译某个密码算法的所需的计算时间或成本费用远远超过信息有用的生命周期或者信息本身的价值时,那算法破译本身就没有意义了。这时也可以认为该算法具有计算安全性。
  \item \textbf{可证明安全性}。算法的安全性可规约为某个经过深入研究的数学难题(如大整数素因子分解、计算离散对数等),数学难题被证明求解困难。不过,当量子计算出现之后,针对目前使用的RSA、DH和ECC等公钥算法的计算安全性不再有理论保证,因此,密码界正在开展抗量子密码研究,以期在量子计算机成为现实攻击工具之前找到新的出路。
\end{enumerate}


Diffie-Hellman 算法的安全性:

偷听者("Eve")可能必须通过求解迪菲-赫尔曼问题来得到 $g^{ab}$,该问题又称为离散对数问题,其求解是非常困难的。

1) 但是如果Alice和Bob使用的随机数生成器不能做到完全随机并且从某种程度上讲是可预测的,那么Eve的工作将简单的多。
还有一个 DH 早期版本的安全性问题是中间人攻击,攻击者 Eve 可以在Alice和Bob通讯的过程中,截获Alice和Bob的消息并伪造假消息,使得Alice计算出来的密钥实际上为Alice和Eve之间协商的密钥,Bob计算出来的密钥实际上是Blob和Eve之间协商的密钥。
密钥交换不能抵御上述攻击,是因为没有对通信的参与方进行认证。
所以 TLS 就引入了认证机制来进行防御。


2) 另外一个安全性问题就是并非所有Diffie-Hellman参数都可以“安全”使用。 Diffie-Hellman的安全性取决于称为离散对数问题的特定数学问题的难度。 如果可以解决一组参数的离散对数问题,则可以提取私钥并破坏协议的安全性。 一般来说,使用的数字越大,解决离散对数问题就越困难。 因此,如果选择较小的DH参数,则会遇到麻烦。
2015年的LogJam和WeakDH攻击表明,许多TLS服务器可能被欺骗使用Diffie-Hellman的小数字,允许攻击者破坏协议的安全性并解密对话。
Diffie-Hellman还要求参数具有某些其他数学属性。 2016年,Antonio Sanso在OpenSSL中发现了一个问题,其中选择的参数缺乏正确的数学属性,导致另一个漏洞。
TLS 1.3采用固定路由,将Diffie-Hellman参数限制为已知安全的参数。 但是,它仍然有几个选择; 只允许一个选项使得在以后发现这些参数不安全的情况下更新TLS非常困难。

\end{enumerate}

\end{document}
