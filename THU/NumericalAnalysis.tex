\chapter{工程硕士数学(a.k.a. 数值分析)}
\label{chap:numerical_analysis}

\chapternote{主要内容是讲解数值问题和求解方程}{工程硕士数学@Winter 2020}

\begin{learningobjectives}
	\item 误差,有效数值
	\item 数值稳定性
\end{learningobjectives}

\dependencies{先修课程:线性代数}

\section{绪论}

\subsection{计算机上的数}

采用\emph{浮点数}表示,尾数和阶数都具有限精度的位数,超过其位数会被\emph{截断}。

\concept{病态问题}:原始数据的微小变化会引起计算结果的巨大变化。

\subsection{误差与有效数字}

误差来源:

\begin{enumerate}
	\item 数学模型
	\item 观测误差
	\item 截断误差
	\item 舍入误差
\end{enumerate}

设 $x, x^*$ 分别为准确值和近似值,对 $\epsilon = |x - x^*| \le \sigma(x^*), \epsilon_r = |x - x^*| / |x^*| \le \sigma_r(x^*) = \sigma(x^*) / |x^*|$,$\epsilon$ 和 $\sigma(x^*)$ 分别为\concept{绝对误差}和\concept{绝对误差界},$\epsilon_r, \sigma_r$ 为\concept{相对误差}和\concept{相对误差界}。

将 $x^*$ 转化为类似于科学技术法的形式:
% (DCMMC): use align* to write multiline equations, and use `&` to state the
% alignment points.
% (DCMMC): use `\ ` to insert spaces in math equations.
\begin{align*}
	&|x - x^*| = \pm 10^k \times 0.a_1a_2 \cdots c_n \cdots \le 0.5 \times 10^{k-n} \\
	&s.t.\ a_1 \ne 0,  1 \le a_i \le 9, a_i \in \N, k \in \Z, n \in \N
\end{align*}
其中 $\Z$ 表示整数集,$n$ 表示\concept{有效数字位数}。

\concept{函数误差}:
\begin{align*}
	|f(x) - f(x_A)| \le |f^\prime (x_A)||x - x_A|
\end{align*}
其中 $x_A$ 是 $x$ 的近似值。

数值计算中常见误差的解决方法:
\begin{enumerate}
	\item 两个相近的数相减:使用分子有理化将减转化为加
	\item 避免大数吃小数
	\item 避免除数的绝对值远小于被除数的绝对值
	\item 简化运算次数
\end{enumerate}

\concept{数值稳定性}:

如果算法的初始值有误差,在运算中误差无限增加,不能控制,则该算法是数值不稳定的,反之则是数值稳定的。

\section{线性代数复习}

对角阵,三角阵的乘法和逆矩阵还是对角阵,三角阵。

正定阵:

矩阵的特征值:方程 $(\lambda I-A)x = \boldsymbol{0}$的非零解 $X\ne\boldsymbol{0}$ 就是 $A$ 的特征向量。

如果特征值均大于 $0$,则矩阵为正定阵。

Guass 消去法解方程的公式 2.3 不需要记.

\section{线性代数方程组的直接解法}

范数是一种距离度量,他就是一个实数,需要满足相应的条件。

矩阵的范数可以由向量范数来定义。

矩阵的平方的特征值就是原来矩阵特征值的平方。

证明等号可以替换为证明 $\le$ 和 $\ge$.

